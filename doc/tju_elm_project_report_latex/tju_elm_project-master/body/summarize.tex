\chapter{问题,解决对策及反思}

\section{浮点数精度}
问题:如果出现3个0.1相加的情况,得到的结果就是0.3000004

解决方案:用字符串表示小数,然后通过字符串相加的算法,得到最终结果

\section{页面显示不全}
问题:有些机型在页面下拉的时候,由于底部有遮挡,没有办法下拉到最下边

解决方案:底部适当留白

反思:站在用户的角度,完善页面布局,给用户最好的体验

\section{历史订单的价格不能随着新的商品价格变化而变化}
问题:当改变数据库一个商品价格的时候,之前已经购买的商品的价格也会变化

解决方案:在原有的订单明细表中,增加一个属性;商品的priceAtThatTime,记住商品在购买时的价格
,当展示订单明细的时候,不用在数据库中去找现在的商品,而是直接在订单明细表中拿到当时的商品价格

反思:当想要保留一个数据的时候,可以直接在数据库中增加属性,在创建的时候,可以直接将值保留进数据库,
在想用的时候直接从数据库中拿到想要的值就可以了


\section{商品详情页的图片不能显示}
问题:商品详情页的图片不能显示,因为图片的url是相对路径,需要将图片的url改为绝对路径

解决方案:在数据库中,将图片的url改为绝对路径,在商品详情页中,通过绝对路径来获取图片

反思:在数据库中,将图片的url改为绝对路径,在商品详情页中,通过绝对路径来获取图片,
这样,在商品详情页中,图片的url就无需再通过服务器来获取了

\section{搜索}
问题:搜索功能,只能搜索到商家,无法搜索到想要的商品

解决方案:强化搜索的sql语句:先将商家与商品按照商家的id进行左连接,并筛选出带有关键词的
商家or商品,将筛选出满足条件的所有商家的id,再显示所有的商家id在(in)刚才筛选出的id中的商家即可

反思:优化sql语句,可以实现更强大的功能

\section{手机号和密码的强度检验}
问题:如果没有检验,用户使用的手机号和密码将会泛滥,失去控制

解决方案:使用正则表达式进行手机号与密码的强度判断

\section{点赞,收藏,评论的存储}

问题:用户的点赞,收藏,评论如何永久保留

解决方案:在数据库中新建点赞,收藏,评论的表结构,并约束好主键外键,设计索引以便更方便的查询数据库中的数据

\section{文心一言的接口}

问题:在javaweb工程中,如何调用文心一言的接口

解决方案:通过查询文心一言api的官网,根据教程与代码,在javaweb工程中,通过HttpClient调用文心一言的接口,并获取返回结果

\section{部署到云端}

问题:如何将本地的javaweb工程部署到云端,并让云端能访问到前端后端以及数据库

解决方案:注册腾讯云账号,开通一个云服务器,因为云服务器上的操作系统是linux,
不方便操作,所以安装宝塔面板,通过数据库部署,html前端部署,java后端部署以及一些配置
完成云端部署,并可以通过公网ip访问到云端的项目