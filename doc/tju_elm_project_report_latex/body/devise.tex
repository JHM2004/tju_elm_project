
\chapter{项目设计文档}

\section{数据库设计}

\subsection{business表}


\begin{longtable}[]{@{}lllll@{}}
\toprule
Field & Null & Key & Default & Extra\tabularnewline
businessId & NO & PRI & NULL & auto\_increment\tabularnewline
phoneNumber & NO & & NULL &\tabularnewline
password & NO & & NULL &\tabularnewline
businessName & NO & & 未命名商家 &\tabularnewline
businessAddress & YES & & NULL &\tabularnewline
businessExplain & YES & & NULL &\tabularnewline
businessImg & YES & & NULL &\tabularnewline
orderTypeId & YES & & NULL &\tabularnewline
starPrice & YES & & 0 &\tabularnewline
deliveryPrice & YES & & 0 &\tabularnewline
remarks & YES & & NULL &\tabularnewline
\bottomrule
\end{longtable}

\subsection{cart表}

\begin{longtable}[]{@{}llllll@{}}
\toprule
Field & Type & Null & Key & Default & Extra\tabularnewline
cartId & int & NO & PRI & NULL & auto\_increment\tabularnewline
foodId & int & NO & & NULL &\tabularnewline
businessId & int & NO & & NULL &\tabularnewline
userId & varchar(20) & NO & & NULL &\tabularnewline
quantity & int & NO & & NULL &\tabularnewline
\bottomrule
\end{longtable}

\subsection{chats表}

\begin{longtable}[]{@{}llllll@{}}
\toprule
Field & Type & Null & Key & Default & Extra\tabularnewline
currentUserId & varchar(50) & NO & & NULL &\tabularnewline
senderUserId & varchar(50) & NO & & NULL &\tabularnewline
receiverUserId & varchar(50) & NO & & NULL &\tabularnewline
message & varchar(255) & YES & & NULL &\tabularnewline
\bottomrule
\end{longtable}

\subsection{deliveryaddress表}

\begin{longtable}[]{@{}llllll@{}}
\toprule
Field & Type & Null & Key & Default & Extra\tabularnewline
daId & int & NO & PRI & NULL & auto\_increment\tabularnewline
contactName & varchar(20) & NO & & NULL &\tabularnewline
contactSex & int & NO & & NULL &\tabularnewline
contactTel & varchar(20) & NO & & NULL &\tabularnewline
address & varchar(100) & NO & & NULL &\tabularnewline
\bottomrule
\end{longtable}

\subsection{favorite表}

\begin{longtable}[]{@{}llllll@{}}
\toprule
Field & Type & Null & Key & Default & Extra\tabularnewline
userId & varchar(20) & NO & & NULL &\tabularnewline
businessId & int & NO & & NULL &\tabularnewline
\bottomrule
\end{longtable}

\subsection{food表}

\begin{longtable}[]{@{}llllll@{}}
\toprule
Field & Type & Null & Key & Default & Extra\tabularnewline
foodId & int & NO & PRI & NULL & auto\_increment\tabularnewline
foodName & varchar(30) & NO & & NULL &\tabularnewline
foodExplain & varchar(30) & NO & & NULL &\tabularnewline
foodImg & mediumtext & NO & & NULL &\tabularnewline
foodPrice & decimal(5,2) & NO & & NULL &\tabularnewline
businessId & int & NO & & NULL &\tabularnewline
remarks & varchar(40) & YES & & NULL &\tabularnewline
\bottomrule
\end{longtable}

\subsection{likes表}

\begin{longtable}[]{@{}llllll@{}}
\toprule
Field & Type & Null & Key & Default & Extra\tabularnewline
userId & varchar(20) & NO & PRI & NULL &\tabularnewline
businessId & int & NO & PRI & NULL &\tabularnewline
\bottomrule
\end{longtable}

\subsection{orderdetailet表}

\begin{longtable}[]{@{}llllll@{}}
\toprule
Field & Type & Null & Key & Default & Extra\tabularnewline
odId & int & NO & PRI & NULL & auto\_increment\tabularnewline
foodName & varchar(40) & YES & & NULL &\tabularnewline
orderId & int & NO & & NULL &\tabularnewline
foodId & int & NO & & NULL &\tabularnewline
quantity & int & NO & & NULL &\tabularnewline
priceAtThatTime & decimal(5,2) & NO & & 0 &\tabularnewline
\bottomrule
\end{longtable}

\subsection{orderdetailet food表} 

\begin{longtable}[]{@{}llllll@{}}
\toprule
Field & Type & Null & Key & Default & Extra\tabularnewline
odId & int & YES & & NULL &\tabularnewline
orderId & int & YES & & NULL &\tabularnewline
foodId & int & YES & & NULL &\tabularnewline
quantity & int & YES & & NULL &\tabularnewline
priceAtThatTime & decimal(5,2) & YES & & NULL &\tabularnewline
foodName & varchar(255) & YES & & NULL &\tabularnewline
foodExplain & varchar(255) & YES & & NULL &\tabularnewline
foodImg & varchar(255) & YES & & NULL &\tabularnewline
businessId & int & YES & & NULL &\tabularnewline
remarks & varchar(255) & YES & & NULL &\tabularnewline
\bottomrule
\end{longtable}

\subsection{orders表}

\begin{longtable}[]{@{}llllll@{}}
\toprule
Field & Type & Null & Key & Default & Extra\tabularnewline
orderId & int & NO & PRI & NULL & auto\_increment\tabularnewline
userId & varchar(20) & NO & & NULL &\tabularnewline
businessId & int & NO & & NULL &\tabularnewline
orderDate & date & YES & & NULL &\tabularnewline
orderTotal & decimal(7,2) unsigned zerofill & NO & & 0 &\tabularnewline
daId & int & YES & & NULL &\tabularnewline
orderState & int & NO & & 0 &\tabularnewline
\bottomrule
\end{longtable}

\subsection{remarks表}

\begin{longtable}[]{@{}llllll@{}}
\toprule
Field & Type & Null & Key & Default & Extra\tabularnewline
remark & varchar(255) & YES & & NULL &\tabularnewline
businessId & int & YES & & NULL &\tabularnewline
remarkDate & date & YES & & NULL &\tabularnewline
userId & varchar(20) & YES & & NULL &\tabularnewline
remarkId & int & NO & PRI & NULL & auto\_increment\tabularnewline
userName & varchar(20) & YES & & NULL &\tabularnewline
\bottomrule
\end{longtable}

\subsection{searchhistory表}

\begin{longtable}[]{@{}llllll@{}}
\toprule
Field & Type & Null & Key & Default & Extra\tabularnewline
userId & varchar(40) & NO & & NULL &\tabularnewline
searchContent & varchar(255) & NO & & NULL &\tabularnewline
\bottomrule
\end{longtable}

\subsection{user表}

\begin{longtable}[]{@{}llllll@{}}
\toprule
Field & Type & Null & Key & Default & Extra\tabularnewline
userId & varchar(20) & NO & PRI & NULL &\tabularnewline
password & varchar(20) & NO & & NULL &\tabularnewline
userName & varchar(20) & NO & MUL & NULL &\tabularnewline
userSex & int & NO & & 1 &\tabularnewline
userImg & mediumtext & YES & & NULL &\tabularnewline
delTag & int & NO & & 1 &\tabularnewline
\bottomrule
\end{longtable}


\section{前端设计}

\subsection{底部导航栏}
    


    首页:返回主界面。

    我的:包含用户的个人中心,如订单、收藏、账户设置等。

    发现:包含发现新商家或新活动的功能。

    订单:查看和管理用户的订单。
\subsection{首页}

    顶部导航栏:

    登录和注册按钮,允许用户进行账户管理。

    天津大学北洋园校区的下拉菜单,用于选择不同的校区或位置。

    搜索栏:

    提供搜索功能,用户可以搜索饿了么平台上的商家或商品名称。

    分类标签:

    早餐美食:可能包含早餐相关的食品选项。

    跑腿代购:提供代购服务,如汉堡披萨、甜品饮品等。

    速食简餐:快速方便的食品选择。

    地方小吃:提供地方特色小吃。

    米粉面馆:提供米粉、面条、包子、粥、炸鸡、炸串等食品。

    品质套餐:提供搭配齐全的套餐选项。

    促销活动:

    立即抢购:可能是限时抢购或特价活动的入口。

    超级会员:提供会员服务,每月享受超值权益。

    商家推荐:

    推荐商家列表,显示商家名称、评分、月销量、配送信息等。

    筛选和排序功能:

    筛选:允许用户根据综合排序、距离最近、销量最高等条件筛选商家。

    排序:提供排序功能,用户可以根据不同的标准对商家进行排序。

    商家详情:

    万家饺子(软件园E18店):显示商家的详细信息,包括评分、月销量、配送费用、起送价、配送距离和预计送达时间。

    蜂鸟专送:表明配送服务由蜂鸟专送提供。
\subsection{登录}

    密码输入区域:

    两个密码输入框,用于输入密码和确认密码。

    操作按钮:

    登录按钮:用于提交登录信息。

    去注册链接:引导新用户注册账户。
\subsection{商家列表} 

    标题:

    商家列表:标明这是商家列表的页面。

    商家信息:

    万家饺子(软件园E18店):商家名称,包括分店信息。

    商品种类:各种饺子,说明商家提供的主要商品。

    小锅饭豆腐馆(全运店):商家名称,包括分店信息。

    商品种类:小锅套餐,说明商家提供的主要商品。

    米村拌饭(浑南店):商家名称,包括分店信息。

    商品种类:拌饭、拌饭套餐、串,说明商家提供的主要商品。

    申记串道(中海康城店):商家名称,包括分店信息。

    商品种类:烤串、炸串,说明商家提供的主要商品。
    \subsection{商家信息}
 

    食品名称:食品的名称,包括烹饪方式。

    食品描述:描述食品的主要食材和特点。

    价格信息:食物的价格

    配送费用:另需配送费,说明除了食品价格外,还需要额外支付的费用。

    起送金额:标明用户下单的最低金额要求。
    \subsection{商单信息}

    订单标题:

    确认订单:标明这是用户需要确认的订单信息。

    配送信息:订单配送至:提示用户订单将被配送到的地址,具体的配送地址,收货人的联系电话。

    商家信息:订单来自的商家名称和分店信息。

    订单明细:用户所点的食品名称和数量,食品的价格,订单的配送费用。

    支付按钮:用户点击此按钮进行支付。

    订单总额:包括食品价格和配送费用的订单总金额。

    \subsection{支付信息}

    支付标题:

    在线支付:标明这是用户进行在线支付的页面。

    订单信息:

    订单信息:提示用户这是他们即将支付的订单详情。

    商家及订单金额:

    订单来自的商家名称和分店信息。

    订单的总金额,可能包括食品价格和配送费用。

    支付方式:

    支付宝:提供支付宝作为支付选项。

    微信支付:提供微信支付作为支付选项。

    支付按钮:用户点击此按钮以确认支付订单。
    \section{后端设计}
    
    \subsection{``我的''页面新增api}

    3.1.1.头像与昵称的获取接口:(原接口)

    UserController / getUserByIdByPass

    参数:userId、password

    返回值:user对象

    3.1.2.修改头像:(新增接口)

    UserController / changeUserAvatar

    参数:userId , base64 (头像的base64编码)

    返回值:int(影响的行数)

    3.1.3.修改昵称:(新增接口)

    UserController / changeUserName

    参数:userId , userName

    返回值:int (影响的行数)

    3.1.4.修改密码:(新增接口)

    UserController / changeUserPassword

    参数:userId , oldPassword , newPassword

    返回值:int (影响的行数)

    3.1.5 UserController/saveUser

    参数:userId , password , userName , userSex , userImg (可选)

    返回值:int (影响的行数)

    功能:向用户表中添加一条记录
    \subsection{评论和展示评论}

    3.2.1 RemarkController / listRemarksByBusinessId

    参数:businessId

    返回值:remark数组

    功能: 根据商家编号查询所属商家的所有评论信息

    3.2.2 RemarkController / saveRemarks

    参数:remark,userId,userName,businessId

    返回值:int (评论编号)

    功能:向评论表中添加一条记录,并返回评论编号

    3.2.3 RemarkController / removeOneRemark

    参数:userName userId businessId remark

    返回值:int 影响的行数

    功能:用户删除在某商家下的一条评论
    \subsection{收藏商家与展示收藏列表}

    3.3.1 FavoriteController / listFavoriteByUserId

    参数:userId

    返回值: businessId数组

    功能: 查询用户收藏的所有商家

    3.3.2 FavoriteController / saveFavoriteBusinessId

    参数:userId , businessId

    返回值:int (影响的行数)

    功能:当前用户收藏该商家(向收藏表中添加一条数据)

    3.3.3 FavoriteController / removeFavoriteBusinessId

    参数:userId , businessId

    返回值:int (影响的行数)

    功能:当前用户取消收藏该商家
    \subsection{点赞}

    3.4.1 LikesController / getLikesBybusinessId

    参数:businessId

    返回值: int (此商家的点赞总数量)

    功能: 查询某商家的点赞总数量

    3.4.2 LikesController / saveLikes

    参数:userId , businessId

    返回值:int (影响的行数)

    功能:用户点赞某一个商家

    3.4.3 LikesController / removeLikes

    参数:userId , businessId

    返回值:int (影响的行数)

    功能:用户取消点赞某一个商家

    3.4.4 LikesController / getLikesByUserIdByBusinessId

    参数:userId , businessId

    返回值:int (0:之前此用户对此商家未点赞

    1:之前此用户对此商家点过赞)

    功能:判断此用户对此商家之前有没有点过赞,

    若点过赞,则再点一次是取消点赞

    若未点过赞,则点一次是点赞
    \subsection{私信}

    3.5.1 ChatsController / removeChatsAllByCurrentUserId

    参数:currentUserId

    返回值:int (影响的行数)

    功能:清空当前用户的私信列表

    3.5.2 ChatsController / removeChatsByTwoUserIdByMessage

    参数:currentUserId , receiverUserId , message

    返回值:int (影响的行数)

    功能:删除当前用户currentUserId的与

    用户receiverUserId的内容为message的消息

    3.5.3 ChatsController / recallChatsByTwoUserIdByMessage

    参数:currentUserId , receiverUserId , message

    返回值:int (影响的行数)

    功能:撤回当前用户currentUserId的发送给

    用户receiverUserId的内容为message的消息

    3.5.4 ChatsController / getChatsByUserId

    参数:currentUserId

    返回值: chats数组

    功能: 查询当前用户的所有私信列表

    3.5.5 ChatsController / saveChats

    参数:senderUserId , receiverUserId , message

    返回值:int (影响的行数)

    功能:新增私信记录到总私信表中
    \subsection{商家登陆}

    3.6.1 BusinessController / saveBusiness

    参数:businessName ,password

    返回值:int (商家编号businessId)

    功能: 商家注册 并返回代表商家的唯一编号作为账号

    3.6.2 BusinessController / updateBusiness

    参数:businessId , businessAddress ,

    businessExplain , businessImg ,

    starPrice , deliveryPrice , orderTypeId

    返回值: int (影响的行数)

    功能:更改(完善)商家的信息

    3.6.3 FoodController / addFood

    参数:businessId , foodName , foodExplain ,

    foodImg , foodPrice

    返回值:int (食品的标号foodId)

    功能:商家可以上架自己的商品

    3.6.4 BusinessController / getBusinessByIdByPass

    参数:businessId , password

    返回值:int (影响的行数)

    功能:商家登录
    \subsection{搜索与搜索历史}

    3.7.1 SearchController / listBusiness

    参数:searchContent , userId (需要更新userId的这条历史搜索)

    返回值:business数组

    功能:根据关键词搜索商家列表

    3.7.2 SearchController / getHistoryByUserId

    参数:userId

    返回值:searchContent

    功能:查询用户上一次的一条搜索记录

    3.7.3 BusinessController / listBusinessByOrderTypeId

    参数:orderTypeId

    返回值:business数组

    功能:根据点餐分类编号查询所属商家信息

    3.7.4 BusinessController / getBusinessById

    参数:businessId

    返回值:business对象

    功能:根据商家编号查询商家信息

    3.7.5 RemarkController / listRemarksByBusinessId

    参数:businessId

    返回值:remark数组

    功能: 根据商家编号查询所属商家的所有评论信息
    \subsection{原项目基础API}

    \protect\hypertarget{2.3.1.business}{}{}3.8.1 business

    3.8.1.1 BusinessController/listBusinessByOrderTypeId

    参数:orderTypeId

    返回值:business数组

    功能:根据点餐分类编号查询所属商家信息

    3.8.1.2 BusinessController/getBusinessById

    参数:businessId

    返回值:business对象

    功能:根据商家编号查询商家信息

    \protect\hypertarget{2.3.2.food}{}{}3.8.2 food

    3.8.2.1. FoodController/listFoodByBusinessId

    参数:businessId返回值:food数组

    功能:根据商家编号查询所属食品信息

    \protect\hypertarget{2.3.3.cart}{}{}3.8.3 cart

    3.8.3.1 CartController/listCart

    参数:userId、businessId(可选)

    返回值:cart数组(多对一:所属商家信息、所属食品信息)功能:根据用户编号查询此用户所有购物车信息

    根据用户编号和商家编号,查询此用户购物车中某个商家的所有购物车信息

    3.8.3.2 CartController/saveCart

    参数:userId、businessId、foodId返回值:int(影响的行数)

    功能:向购物车表中添加一条记录

    3.8.3.3 CartController/updateCart

    参数:userId、businessId、foodId、quantity返回值:int(影响的行数)

    功能:根据用户编号、商家编号、食品编号更新数量

    3.8.3.4 CartController/removeCart

    参数:userId、businessId、foodId(可选)返回值:int(影响的行数)

    功能:根据用户编号、商家编号、食品编号删除购物车表中的一条食品记录根据用户编号、商家编号删除购物车表中的多条条记录

    \protect\hypertarget{2.3.4.deliveryAddress}{}{}3.8.4 deliveryAddress

    3.8.4.1 DeliveryAddressController/listDeliveryAddressByUserId

    参数:userId

    返回值:deliveryAddress数组

    功能:根据用户编号查询所属送货地址

    3.8.4.2 DeliveryAddressController/getDeliveryAddressById

    参数:daId

    返回值:deliveryAddress对象

    功能:根据送货地址编号查询送货地址

    3.8.4.3 DeliveryAddressController/saveDeliveryAddress

    参数:contactName、contactSex、contactTel、address、userId返回值:int(影响的行数)

    功能:向送货地址表中添加一条记录

    3.8.4.4 DeliveryAddressController/updateDeliveryAddress

    参数:daId、contactName、contactSex、contactTel、address、userId返回值:int(影响的行数)

    功能:根据送货地址编号更新送货地址信息

    3.8.4.5 DeliveryAddressController/removeDeliveryAddress

    参数:daId

    返回值:int(影响的行数)

    功能:根据送货地址编号删除一条记录

    \protect\hypertarget{2.3.5.orders}{}{}3.8.5 orders

    3.8.5.1 OrdersController/createOrders

    参数:userId、businessId、daId、orderTotal返回值:int(订单编号)

    功能:根据用户编号、商家编号、订单总金额、送货地址编号向订单表中添加一条记录,并获取自动生成的订单编号,

    然后根据用户编号、商家编号从购物车表中查询所有数据,批量添加到订单明细表中,然后根据用户编号、商家编号删除购物车表中的数据。

    3.8.5.2 OrdersController/getOrdersById

    参数:orderId

    返回值:orders对象(包括多对一:商家信息; 一对多:订单明细信息)

    功能:根据订单编号查询订单信息,包括所属商家信息,和此订单的所有订单明细信息

    3.8.5.3 OrdersController/listOrdersByUserId

    参数:userId

    返回值:orders数组(包括多对一:商家信息;
    一对多:订单明细信息)功能:根据用户编号查询此用户的所有订单信息

    新增:

    3.8.5.4.OrdersController / payOk

    参数:orderId

    返回值:int (影响的行数)

    功能:改变订单属性,将未支付0转成已支付1

    \protect\hypertarget{2.3.6.user}{}{}3.8.6 user

    3.8.6.1 UserController/getUserByIdByPass

    参数:userId、password返回值:user对象

    功能:根据用户编号与密码查询用户信息

    3.8.6.2 UserController/getUserById

    参数:userId

    返回值:int(返回行数)

    功能:根据用户编号查询用户表返回的行数

